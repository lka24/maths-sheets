
\documentclass{report}
\usepackage[T1]{fontenc}
\usepackage{graphicx} % Required for inserting images
\usepackage[a4paper, margin=0.75in]{geometry}
\usepackage{amsmath}
\usepackage{tikz}
\renewcommand{\familydefault}{\sfdefault}
\usepackage{pgfplots}
\usepackage{amssymb}
\usepackage{cancel}
\usepackage{verbatim}
\usepackage{nameref}
\usepackage{paracol}
\usepackage{tikz-cd}
\usepackage{derivative}
\usepackage{url}
\usepackage{amsthm}
\usepackage{listings}
\usepackage{xfrac}
\usepackage{multicol}
\usepackage{caption}
\usepackage{tcolorbox}
\usepackage{soul}
\usepackage{fancyhdr}
\pagestyle{fancy}
\usetikzlibrary{decorations.pathreplacing}
\usetikzlibrary{decorations.markings}
\usetikzlibrary{shapes.misc}
\usepackage[english]{babel}
\globalcounter{figure}

\lstdefinelanguage{R}{
  keywords={if, else, repeat, while, function, for, in, next, break},
  otherkeywords={TRUE, FALSE, NULL, Inf, NaN, NA},
  sensitive=true,
  morecomment=[l]{\#},
  morestring=[b]",
  morestring=[b]',
}

\lstdefinelanguage{Wolfram}{
  keywords={If, Else, For, While, Do, Return, Module, Block, Set, SetDelayed, Function, Table, Map, Apply, Thread, Replace, ReplaceAll},
  otherkeywords={True, False, Null, Infinity, Indeterminate},
  sensitive=true,
  morecomment=[s]{(*}{*)},
  morestring=[b]",
  keywordstyle=\color{cyan},      % Cyan for control structures
  identifierstyle=\color{white},  % White for custom variables
  emph={Sin, Cos, Tan, Log, Exp, Sqrt, Cosh, Sinh, Tanh, ArcSin, ArcCos, ArcTan, ArcSinh, ArcCosh, ArcTanh},
  emphstyle=\color{magenta},       % Magenta for built-in math functions
  commentstyle=\color{lightgray},  % Light gray for comments
  stringstyle=\color{orange},      % Orange for strings
  numberstyle=\color{yellow},      % Yellow for numbers
  literate={^}{{\textasciicircum}}1,
  xleftmargin=0pt,               % Ensure no extra indentation
}

\lstset{
  basicstyle=\ttfamily\color{white},      % White text
  keywordstyle=\color{cyan},               % Cyan keywords
  commentstyle=\color{gray},               % Gray comments
  stringstyle=\color{orange},              % Orange strings
  numberstyle=\tiny\color{lightgray},      % Light gray line numbers
  showstringspaces=false,
  breaklines=true,
  backgroundcolor=\color{black}           % Black background
}

\let\Subsectionmark\subsectionmark
\def\Subsectionname{}
\def\subsectionmark#1{\def\Subsectionname{#1}\Subsectionmark{#1}}
\fancyhead[R]{\Subsectionname}

\newtcolorbox{example}[1][]{%
    colframe=black, % Frame color
    colback=gray!7, % Background color
    fonttitle=\bfseries, % Title font styling
    title=Example
    \ifx&#1& % If first argument is empty, do nothing
    \else
        ~(#1)%
    \fi,
    sharp corners, % Box styling
}

\newtcolorbox{warning}[1][]{%
    colframe=red, % Frame color
    colback=red!7,        % Background color
    fonttitle=\bfseries,    % Title font styling
    title=Warning!%
    \ifx&#1&% Check if #1 is empty
    \else
        ~(#1)%
    \fi,
    sharp corners,          % Box styling
}

% Define a custom code block environment with verbatim
\newtcolorbox[auto counter, number within=chapter]{codeblock}[2][]{
  colback=black,
  colframe=black,
  fontupper=\ttfamily\color{white},
  fonttitle=\bfseries,
  title=Code Block \thetcbcounter: #2,
  sharp corners,
  #1
}

\newenvironment{solution}
    {\renewcommand\qedsymbol{}\begin{proof}[Solution]}
    {\end{proof}}


\newcommand{\Real}{\mathbb{R}}
\newcommand{\Expect}[1]{\mathbb{E}\left[#1\right]}
\newcommand{\Var}[1]{\mathrm{Var}\left[#1\right]}
\newcommand{\Prob}[1]{\mathbb{P}\left(#1\right)}
\newcommand{\Cov}[2]{\mathrm{Cov}\left(#1,#2\right)}
\newcommand{\Rat}{\mathbb{Q}}
\newcommand{\Img}[1]{\mathrm{im}\left(#1\right)}
\newcommand{\Ker}[1]{\mathrm{ker}\left(#1\right)}
\newcommand{\ftr}[1]{\mathcal{F}\{#1\}}
\newcommand{\invftr}[1]{\mathcal{F}^{-1} \left\{ #1 \right\}}
\newcommand{\ftrcos}[1]{\mathcal{F}_c \left\{#1 \right\}}
\newcommand{\ftrsin}[1]{\mathcal{F}_s \left\{#1 \right\}}
\newcommand{\supp}[1]{\mathrm{supp}\left(#1\right)}
\newcommand{\Nat}{\mathbb{N}}
\newcommand{\Int}{\mathbb{Z}}
\newcommand{\di}[1]{\mathop{\mathrm{d}#1}}
\newcommand{\deriv}[2]{\frac{\mathrm{d} #1}{\mathrm{d}#2}}
\newcommand{\kderiv}[3]{\frac{\mathrm{d}^{#3}#1}{\mathrm{d}#2^{#3}}}
\newcommand{\set}[1]{\{ #1 \}}
\newcommand{\dls}[2]{\mathrm{L} \left( #1, #2 \right)}
\newcommand{\dus}[2]{\mathrm{U} \left( #1, #2 \right)}
\newcommand{\like}[2]{\mathrm{L}\left(#1 \vert #2 \right)}
\newcommand{\abs}[1]{\left| #1 \right|}
\newcommand{\Span}[1]{\mathrm{Span}\left( #1 \right)}
\newcommand{\irv}[1]{\mathbb{I}_{#1}}
\newcommand{\ord}[1]{\mathrm{ord}\left(#1\right)}
\newcommand{\cycg}[1]{\left\langle #1 \right\rangle}
\newcommand{\idx}[2]{\left| #1 : #2 \right|}
\newcommand{\rss}{\mathrm{RSS}}
\renewcommand{\vec}[1]{\mathbf{#1}}
\renewcommand{\phi}{\varphi}
\newtheorem{theorem}{Theorem}[section]
\newtheorem{lemma}[theorem]{Lemma}
\newtheorem{definition}[theorem]{Definition}

\tcolorboxenvironment{specialtheorem}{
    colback=yellow!5,
    colframe=orange!80!black,
    before skip=10pt, after skip=10pt,
    boxrule=1pt,
    sharp corners,
    fonttitle=\bfseries,
}

\newtheorem{specialtheorem}[theorem]{Special Theorem}

\pgfplotsset{compat=newest}
\usepgfplotslibrary{colormaps}
\usetikzlibrary{patterns}

\title{Cayley-Hamilton Proof}
\author{Lorcan Addison}
\date{2025}



\begin{document}
        \begin{theorem}
            Suppose $X$ and $Y$ are independent random variables with
            \[f_{X}(x) = \frac{1}{2\pi} e^{-\frac{1}{2} \left(x^2+y^2\right)} \quad \text{for } x \in \Real\]
            and $Y\sim \chi_{\nu}^2$, that is 
            \[f_Y(y) = \frac{1}{2^{\sfrac{\nu}{2}} \Gamma(\sfrac{\nu}{2})} y^{\sfrac{\nu}{2}-1} e^{-\sfrac{y}{2}} \]
            Then the random variable $T$ defined by
            \[ T := \frac{X}{\sqrt{\sfrac{Y}{\nu}}} \]
            follows Student's $t$-distribution with $\nu$ degrees of freedom.
        \end{theorem}
        \begin{proof}
            Let $T$ be defined as above and define $Z := \sqrt{\sfrac{Y}{\nu}}$.
            Then we have $x(t, z) = tz$ and $y(t, z) = \nu z^2$. We also compute the partial derivatives as
            \[\pdv{x}{t} = z \quad \pdv{x}{z} = t \quad \pdv{y}{t} = 0 \quad \pdv{y}{z} = 2\nu z\]
            Recalling the formula for the transformation of random variables, we can write
            \[f_{T, Z} (t, z) = f_{X, Y} (x(t, z), y(t, z)) \abs{J(t, z)}\]
            where $J(t, z)$ is the Jacobian matrix.
            To start off, we compute the Jacobian as
            \[ \begin{vmatrix} z & t \\ 0 & 2\nu z \end{vmatrix} = 2 \nu z^2 \]
            and, noting that $\nu \in \Nat$, we conclude $\abs{J} = 2 \nu z^2$.

            Therefore,
            \begin{align*}
            f_{T, Z} &= f_{X} (x(t,z)) f_Y (y(t,z)) \abs{J} && \text{(by independence)} \\
            &= \frac{\frac{1}{\sqrt{2\pi}} e^{-\frac{1}{2} t^2 z^2} (\nu z^2)^{\sfrac{\nu}{2}-1} e^{-\frac{1}{2} \nu z^2} (2 \nu z^2)}{
                2^{\sfrac{\nu}{2}} \Gamma(\sfrac{\nu}{2})
            } \\
            &= \frac{
                \nu^{\sfrac{\nu}{2}} z^{\nu} e^{-\frac{1}{2} t^2 z^2} e^{-\frac{1}{2} \nu z^2}
            }{ 2^{\frac{1}{2} (1 - \nu)} \Gamma(\sfrac{\nu}{2}) \sqrt{\pi} }
            \end{align*}
            To obtain the pdf of $T$ we compute the marginal. Note that $z$ ranges from $0$ to $+\infty$.
            \[ f_T(t) = \frac{\nu^{\sfrac{\nu}{2}}}{2^{\frac{1}{2} (1 - \nu)} \Gamma(\sfrac{\nu}{2}) \sqrt{\pi}}
            \underbrace{\int_0^{\infty} z^{\nu} e^{\frac{1}{2} \left(z^2 (t^2 + \nu) \right)} \odif{z}}_{I} \]
            To evaluate $I$ we change the variable to $\gamma := \frac{1}{2} (z^2 (t^2 + \nu))$, so that
            \[ z = \left(\frac{2\gamma}{t^2 + \nu}\right)^{\sfrac{1}{2}} \quad \text{and} \quad \odif{z} = \frac{\odif{\gamma}}{z (t^2 + \nu)}\]
            Therefore
            \begin{align*}
            I &= \int_0^{\infty} \left(\frac{2\gamma}{t^2 + \nu}\right)^{\sfrac{\nu}{2}} e^{-\gamma} \frac{\odif{\gamma}}{
                \left(\frac{2\gamma}{t^2 + \nu}\right)^{\sfrac{1}{2}} (t^2 + \nu)
            } \\
            &= \int_0^{\infty} \left(\frac{2\gamma}{t^2 + \nu}\right)^{\sfrac{\nu}{2}} e^{-\gamma} \frac{1}{
                \sqrt{2} \gamma^{\sfrac{1}{2}} (t^2 + \nu)
            } \odif{\gamma} \\
            &= \sqrt{2}^{\nu-1} (t^2 + \nu)^{-\frac{\nu+1}{2}} \int_0^{\infty} \gamma^{\frac{\nu-1}{2}}
            e^{-\gamma} \odif{\gamma} \\
            &= \sqrt{2}^{\nu-1} (t^2 + \nu)^{-\frac{\nu+1}{2}} \int_0^{\infty} \gamma^{\frac{\nu+1}{2}-1}
            e^{-\gamma} \odif{\gamma} = \sqrt{2}^{\nu-1} (t^2 + \nu)^{-\frac{\nu+1}{2}} \Gamma\left(\frac{\nu+1}{2}\right) \\
            \end{align*}

            Putting this together,
            \begin{align*}
            f_T(t) &= \frac{\cancel{2^{\frac{1}{2}(\nu - 1)}} \nu^{\sfrac{\nu}{2}} (t^2 + \nu)^{\frac{1}{2} (\nu + 1)}}{
                \cancel{2^{\frac{1}{2} (1 - \nu)}} \Gamma(\sfrac{\nu}{2}) \sqrt{\pi}
            } \Gamma\left(\frac{\nu+1}{2}\right) \\
            &= \frac{ \cancel{\nu^{\sfrac{\nu}{2}}} \left(\frac{t^2}{\nu} + 1\right)^{\frac{1}{2} (\nu + 1)}}{
                \Gamma(\sfrac{\nu}{2}) \sqrt{\pi} \cancel{\nu^{\sfrac{\nu}{2}}} \sqrt{\nu}
            } \Gamma\left(\frac{\nu+1}{2}\right) \\
            &= \frac{\Gamma \left(\frac{\nu+1}{2}\right)}{\sqrt{\nu \pi} \Gamma(\sfrac{\nu}{2})}
            \left(\frac{t^2}{\nu} + 1\right)^{\frac{\nu+1}{2}}
            \end{align*}
            
            This is exactly the pdf of Student's $t$-distribution with $\nu$ degrees of freedom.
        \end{proof} 
\end{document}